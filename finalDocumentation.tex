\documentclass[12pt]{article}
\usepackage[utf8]{inputenc}
\usepackage{titling}
\title{Team Star Project \\ \large Final Documentation}
\author{Mubarak Ayantayo, Tim Hunt, Josh Adams, and Jack}
\date{12/6/2024}

\begin{document}

\maketitle


\section{Introduction}
\subsection{\ \ Our client provided us with a solid outline of their needs and expectations for this project, with their being a heavy emphasis on functionality and workability rather than just purely aesthetics. Going from the core points and directions our client provided us, they clearly put an emphasis on player equality in regards to the beginning of the game, as well as heavy emphasis on player and server management. Along with this baseline functionality, they asked for us to put together, they also wanted us to create and add a form of bot to the game to simulate another player, potentially utilizing some form of chat-bot or online artificial intelligence to help streamline the process. Our software solution itself was centered around creating a top down functionality of the code to provide clearly divisible and testable functions to determine individual functionalities and efforts where possible, and to attempt to integrate this clearly testable framework into a UI utilizing firebase as a hosting platform and database. As for the rest of our documentation they can be found under their corresponding sections. The outline of the rest of our methods and solutions can be found in the order and under the sections as follows: Requirements-Functional, Requirements-Nonfunctional, Iterations, Architecture, Testing, and Reflection.}

\newpage
\section{Requirements}
%[[This section outlines the requirements for the software, primarily via user stories. Non-functional requirements are included separately.]]


\subsection{User Roles}
\ \ The types of users for our app can be broken down into three basic types, Users, Guests, and Admins. Two of these User Roles have been further developed and cemented with very specific permissions and allowances. Users and Guests only have permissions to open and join lobbies, whereas admins have permissions to moderate the game and the users to troubleshoot their issues and help keep the game running smoothly.
%[[ Describe the types of users of your App. What is their technical expertise, what security level access should they have, etc.The purpose of this list is to help you ensure you are meeting the clients needs in your requirements.The names you use here to describe roles should match what you use in your user stories and \emph{reflect the clients vocabulary.}

\subsubsection{User: to login and play the game with stat trackers}
\subsubsection{Guest: to play the game}
\subsubsection{Admin: to login and moderate other players accounts and games}

%[[ Copy this subsubsection for each user type. Replace ``User type'' with the name of that user type, such as ``admin'', etc.]]



\subsection{Functional Requirement User Stories and Tasks}

%[[ This section lists all your user stories, \emph{organized} by user type and for each user type organized by FIX ME [explain here the organization scheme you chose -- by priority? By points? By something else? ]. ]] For each user type create a subsubsection of the same format entitled using the user type. ]]


\subsubsection{User}
%[[ Complete the following for each user story.Put a blank line between user stories]]
\textbf{ID: Play BS (CRUD 8 points)}
\begin{enumerate}
    \item Priority: High
    \item \textbf{I want to play BS}
    \item \textbf{So that I can play BS with other people}
    \item \textbf{Definition of Done: When a game can be played to the natural win conditions}
    \item \textbf{Depends On: Other players}
    \item Tasks:
    \begin{enumerate}
    % For each task write the ID, :, then title. Then describe the task.
        \item Task ID: 1, Title: Deal Cards, Description: Deal Cards according to the total amount of players in the lobby
        \item Task ID: 2, Title: Play cards, Description: allow the players to play cards and progress the game
        \item Task ID: 3, Title: Call BS, Description: Allow the players to call BS on another player in the game based on who played last
        \item Task ID: 4, Title: Winners and Losers, Description: create the win condition of having no cards and be declared the winner or loser
        %[[ ID: Title. Description.]]
    \end{enumerate}
\end{enumerate}
\textbf{ID: Create and Join Lobbies (CRUD 8 points)}
\begin{enumerate}
    \item Priority: High
    \item \textbf{I want to create or join someone elses lobby}
    \item \textbf{So that I play with other players}
    \item \textbf{Definition of Done: When a player can create or join a lobby}
    \item \textbf{Depends On: Number of Lobbies and Players}
    \item Tasks:
    \begin{enumerate}
    % For each task write the ID, :, then title. Then describe the task.
        \item Task ID: 5, Title: Lobby Creation, Description: Create an online lobby capable of housing players to play BS
        \item Task ID: 6, Title: Join Lobby, Description: allow players to join previously established lobbies
    \end{enumerate}
\end{enumerate}
\textbf{ID: Public and Private Messaging (CRUD 8 points)}
\begin{enumerate}
    \item Priority: High
    \item \textbf{I want to message in a lobby or privately message someone}
    \item \textbf{So that I can talk with other users}
    \item \textbf{Definition of Done: When users can be messaged both privately and publicly}
    \item \textbf{Depends On: Other players}
    \item Tasks:
    \begin{enumerate}
    % For each task write the ID, :, then title. Then describe the task.
        \item Task ID: 7, Title: Public messaging, Description: Messages all players contained within a lobby
        \item Task ID: 8, Title: Private messaging, Description: Privately messages a single player of choice
    \end{enumerate}
\end{enumerate}
\textbf{ID: User Profiles and Login (CRUD 8 points)}
\begin{enumerate}
    \item Priority: High
    \item \textbf{I want to login and have a username and profile}
    \item \textbf{So that I can see my statistics and manage how I am viewed}
    \item \textbf{Definition of Done: When I can login, logout, and see my lifetime statistics for the game}
    \item \textbf{Depends On: Playing games}
    \item Tasks:
    \begin{enumerate}
    % For each task write the ID, :, then title. Then describe the task.
        \item Task ID: 9, Title: Login, Description: Login to the profile and save user data such as profile name and password
        \item Task ID: 10, Title: Logout, Description: Logout of the game and required to log back in to access the profile
        \item Task ID: 11, Title: Record and Display all User Lifetime info, Description: Allow the user to see all relevant statistics to what they've done in games
    \end{enumerate}
\end{enumerate}
\textbf{ID: Friends List (CRUD 8 points)}
\begin{enumerate}
    \item Priority: High
    \item \textbf{I want to add and remove friends from my friends list}
    \item \textbf{So that I can moderate my friends list}
    \item \textbf{Definition of Done: When I can view, add, and remove friends from my friends list}
    \item \textbf{Depends On: Other players}
    \item Tasks:
    \begin{enumerate}
    % For each task write the ID, :, then title. Then describe the task.
        \item Task ID: 12, Title: Add friends, Description: Add friends to the friends list and view them
        \item Task ID: 13, Title: Remove Friends, Description: Remove friends from friends list 
    \end{enumerate}
\end{enumerate}
\subsection{Non-Functional Requirements}
\subsubsection{User Interface}
\textbf{ID: 14 Title: Appearance of the Cards and Table}
\begin{enumerate}
    \item Description: Looks futuristic and has a purple emphasis
    \item Affects:  How the game and UI is stylized
\end{enumerate}

\section{Iterations}
\subsection{Sprint 1}
\subsubsection{Plan}
\ \ Create baseline code from which to work from, primarily centered around game logic, basic database initialization, UI design and planning
\subsubsection{Activities}
Tim Hunt: Working on the bones of the game logic through the deck, lobby, and player information \newline
Mubarak Ayantayo: Working through friends list and getting the database up and running \newline
Josh: getting the database up and running \newline
Jake:
\subsubsection{Retrospective}
Tim Hunt: Accomplished creating the groundwork from which to build the game logic and data collection, used JUNIT, only person to have testing \newline
Mubarak Ayantayo: got the friends list working \newline
Josh: got the database up \newline
Jake: 
\subsection{Sprint 2}
\subsubsection{Plan}
\ \ Improve the previously established foundations and try to bring code closer to integration with a newly created UI while looking at multiplayer functionality
\subsubsection{Activities}
Tim Hunt: Continue progress on the game logic, further developing towards a playable gamestate and refactoring some game logic to be more top down, alongside extensive testing \newline
Mubarak Ayantayo: Refactoring, database rules, and firestore collections creation to make friends functionality working. Completed sending of friend requests and created the first workable form of UI and the associated login functions as well as the login page information surrounding it. Acclimation to testing frameworks and tools. \newline
Josh: Working on Database \newline
Jake: Looking at Hosting stuff \newline 
\subsubsection{Retrospective}
Tim Hunt: Made substantial progress in refactoring and had fully tested and functional code \newline
Mubarak Ayantayo: Created the first form of the user interface and created some login functions along with it from older code. Did major refactoring and decent progress to testing. first instance of multi player interaction created. \newline
Josh: Working on chat \newline
Jake:
\subsection{Sprint 3}
\subsubsection{Plan}
\ \ Finish all game logic, integrate with UI, resolve lack of testing not associated with game logic
\subsubsection{Activities}
Tim Hunt: Complete all game logic and testing, create a playable game-state and a working game with extensive testing \newline
 Mubarak Ayantayo: Working on integrating the UI with the working gamestate. Made: viewing friend requests, accepting friends, removal of friends, messaging friends, messaging public. creating lobbies. Made progress  joining lobbies, improved the database user fields \newline
Josh: Worked on messages within this sprint. Got closer to getting friends working while attempting to refactor a bit \newline
Jake: ?
\subsubsection{Retrospective}
Tim Hunt: Fully completed game  logic and testing, created a fully functional game which can be played to completion with no outside assistance \newline
Mubarak Ayantayo: Was injured severely and unable to continue work on pace, but still made substantial strides toward goals. Completed  josh and jake tasks and still working on integrating the UI with the game logic in Java \newline
Josh: Refactoring could have been a bit better when coming to my stories but ran out of time \newline
Jake: ?
\section{Architecture}
\ \ The architecture layer from UI to Java is structured to go through HTTP code calling the java code from Firebase, and from the java code side of it it is made from a top down approach focused on having small scale functions being integrated into larger functions letting problems be more easily diagnosed and moderated. The most import and only testing we had was using JUnit on the Java code.
\section{Testing}
\ \ Testing was done exclusively in java code by Tim Hunt, all of which was done utilizing Junit which proved invaluable for testing functions. Testing was used to test every single function of every single class, thoroughly ensuring it was functional. Testing consistently and fully went extremely well. The problem is there is no testing done outside the java side of the code.
\section{Reflection}
\ \ We would not have used firebase at all, we would have used a servlet and have designated clear roles for people who struggled to contribute. The main barriers were integrating completed code with the UI and associated older code. The main facilitators were Tim Hunt and Mubarak Ayantayo. The team was attempted to be organized, but elements within the group prevented this. Tim Hunt was the designated leader, who handled most coordination efforts as well as all documentation, testing, game logic, and most functional code. There was also an attempt by him to keep group members focused and on task. The team was not well organized. A summary of what we have learned about the SE process is that no amount of effort of a single member can make up for a groups worth of work, as well as when designing applications, comprehensive evaluation and planning for major design elements is absolutely critical to the success of an application.

\section{AI Impact}
Tim Hunt: did not use AI as it would slow down and convolute the code and weaken the understanding of the code \newline
Mubarak Ayantayo: Used AI for organization, arrangement, suggestions. manually debugged AI errors, fixed interdependency and accessing the right files to import. AI was a very helpful tool.   \newline
Josh: Used AI exclusively for debugging but that's it. \newline
Jake:


\end{document}